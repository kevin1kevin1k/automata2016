\documentclass{article}
\usepackage
[
    a4paper,
    top=2cm,
    bottom=3cm,
    left=2cm,
    right=2cm
]{geometry}
\usepackage{amssymb}
\usepackage{amsmath}
\usepackage{verbatim}
\usepackage{xeCJK}
\setCJKmainfont{微軟正黑體}
\renewcommand{\thesection}{(\arabic{section})}
\renewcommand{\thesubsection}{L\arabic{subsection}}
\newcommand{\mg}[1][G]{\ifmmode \mathcal{#1} \else $\mathcal{#1}$ \fi}


\title{自動機與形式語言 Homework 4}
\author{Class 02 B03902086 李鈺昇}
\date{}

\begin{document}
    \maketitle
    
    \section{}
        \subsection{}
            Set $q_0 = q_{acc}$ so that the Turing machine directly accepts any input.
        
        \subsection{}
            Set $q_0 = q_{rej}$ so that the Turing machine directly rejects any input.
    
        \subsection{}
            First scan through the input to ensure (1) it's non-empty, (2) it only contains $0$ or $1$, otherwise reject. Then check the rightmost symbol of the input: If it is a $0$, accept; otherwise, reject.

        \subsection{}
            Use a 3-tape (containing tape 1, 2, 3) TM to decide L4. Assume that on input $w$, tape 1 contains $w$, and other tapes are empty.
            \begin{enumerate}
                \item On input $w$, scan through tape 1 to ensure $w \in \{0, 1\}^* - \{\epsilon\}$, otherwise reject.
                \item Assume $w = \lfloor n \rfloor$, the binary representation of $n$. If $n = 1$, reject.
                \item Put $\lfloor 2 \rfloor$ on tape 2.
                \item Let $i$ be the number on tape 2 now. If $i = n$, accept; otherwise, do the following to check if $n \mod i = 0$:
                \begin{itemize}
                    \item Copy $\lfloor n \rfloor$ to tape 3.
                    \item While the number on tape 3 is non-negative, subtract $i$ from tape 3.
                    \item After the loop, it's clear that $n \mod i = 0$ iff the number on tape 3 is $0$.
                \end{itemize}
                \item If $n \mod i = 0$, reject; otherwise, add $1$ to the number on tape 2 and go back to step 4.
            \end{enumerate}
        
    \section{}
        Let $L_1 = \{ 0^n \mid n \geq 0 \}$, $L_2 = \{ 1^n \mid n \geq 0 \}$. Since $L_1$ and $L_2$ can be generated by regex ($0^*$ and $1^*$, respectively), they are regular and thus decidable. Let $M_1$ be the decider of $L_1$, $M_2$ be the decider of $L_2$. \\
        Now if there is a parallel universe, $M_1$ decides $L$; otherwise $M_2$ decides $L$. Only one condition would be true, and in either case we know there \textbf{exists} a TM that decides $L$. So $L$ is decidable.
    
    \section{}
        We can first make the following table by definition: \\ \\
        \begin{tabular}{c|c|c|c|c}
             & $\exists$ acc run, $\nexists$ rej run & $\nexists$ acc run, $\exists$ rej run & $\exists$ acc run, $\exists$ rej run & $\nexists$ acc run, $\nexists$ rej run \\ \hline
            Original NTM & accept & reject & accept & reject \\ \hline
            Bob's NTM & accept & reject & ? & ?
        \end{tabular} \\ \newpage
        
        For a TM \mg[M] and an input $w$,
        \begin{itemize}
            \item On one hand, if there are both accepting run(s) and rejecting run(s), then in the original version \mg[M] accepts $w$, while in Bob's version \mg[M] does not accept $w$. The behavior is not even undefined. \\
            \item On the other hand, if there is neither accepting run nor rejecting run, then in the original version \mg[M] rejects $w$, while in Bob's version \mg[M] does not reject $w$. The behavior is not even undefined.
        \end{itemize}
        
        Hence Bob's definition may lead to some confusion in the above situations.
    
    \section{}
        Restate this problem as $\textsf{CFL-DFA-Equality} = \{ \lfloor \mg, \mg[A] \rfloor \mid \mg \textrm{ is CFG, } \mg[A] \textrm{ is DFA, and } L(\mg) = L(\mg[A]) \}$. \\
        First we know that $\textsf{CFL-Universality} = \{ \lfloor \mg \rfloor \mid \mg = \langle \Sigma, V, R, S \rangle \textrm{ is CFG and } L(\mg) = \Sigma^* \}$ is undecidable. \\
        Let $\mg[A]_0$ be the trivial DFA that accepts all strings, i.e. $L(\mg[A]_0) = \Sigma^*$. \\ \\
        Now define the computable function $F: \Sigma^* \rightarrow \Sigma^*$ such that $F: \lfloor \mg \rfloor \mapsto \lfloor \mg, \mg[A]_0 \rfloor$.
        Since $\lfloor \mg \rfloor \in \textsf{CFL-Universality}$ if and only if $L(\mg) = \Sigma^* = L(\mg[A]_0)$ if and only if $\lfloor \mg, \mg[A]_0 \rfloor \in \textsf{CFL-DFA-Equality}$, we know $\textsf{CFL-Universality} \leq_m \textsf{CFL-DFA-Equality}$.
        It follows that $\textsf{CFL-DFA-Equality}$ is undecidable since $\textsf{CFL-Universality}$ is undecidable.
        
    \section{}
        Restate this problem as $\textsf{CFL-Complement} = \{ \lfloor \mg \rfloor \mid \mg = \langle \Sigma, V, R, S \rangle \textrm{ is CFG and } \Sigma^* - L(\mg) \textrm{is CFL} \}$. \\
        From (4) we know that $\textsf{CFL-DFA-Equality} = \{ \lfloor \mg, \mg[A] \rfloor \mid \mg \textrm{ is CFG, } \mg[A] \textrm{ is DFA, and } L(\mg) = L(\mg[A]) \}$ is undecidable.
        Since DFA can be trivially transformed to deterministic pushdown automaton (DPDA), it follows that $\textsf{CFL-DPDA-Equality} = \{ \lfloor \mg, \mg[D] \rfloor \mid \mg \textrm{ is CFG, } \mg[D] \textrm{ is DPDA, and } L(\mg) = L(\mg[D]) \}$ is undecidable. \\ \\
        Suppose $\textsf{CFL-Complement}$ is decidable by TM \mg[M]. Then we can use the following procedure to decide $\textsf{CFL-DPDA-Equality}$: \\
        \begin{itemize}
            \item Given CFG $\mg$ and DPDA $\mg[D]$, run \mg[M] on $\lfloor \mg \rfloor$.
            \item If the output of \mg[M] is false, then $L(\mg)$ is not DCFL (as shown in the textbook). Then $L(\mg)$ cannot be equal to $L(\mg[D])$, so output false. Otherwise, if the output of \mg[M] is true, then convert \mg to a DCFG \mg[G_D].
            Since the equality problem of DCFL, $\textsf{DCFL-Equality} = \{ \lfloor L_1, L_2 \rfloor \mid L_1, L_2 \textrm{ are DCFL, and } L_1 = L_2 \}$, is decidable, say by TM \mg[N], we can output the result of \mg[N] on $\lfloor L(\mg[D]), L(\mg[G_D]) \rfloor$.
        \end{itemize}
        
        Since $\textsf{CFL-DPDA-Equality}$ is undecidable, we know $\textsf{CFL-Complement}$ is undecidable.
\end{document}